\documentclass[12pt]{article}
\usepackage{amsmath}
\usepackage{amssymb}
\usepackage{amsthm}
\usepackage{enumerate}
\providecommand{\abs}[1]{\lvert#1\rvert}
\providecommand{\norm}[1]{\lVert#1\rVert}

\newtheorem{thm}{Theorem}
\newtheorem{lemma}[thm]{Lemma}
\newtheorem{fact}[thm]{Fact}
\newtheorem{cor}[thm]{Corollary}
\newtheorem{eg}{Example}
\newtheorem{thought}{Thought}
\newtheorem{ex}{Exercise}
\newtheorem{defi}{Definition}
\theoremstyle{definition}
\newtheorem{sol}{Solution}

\newcommand{\ov}{\overline}
\newcommand{\cb}{{\cal B}}
\newcommand{\cc}{{\cal C}}
\newcommand{\cd}{{\cal D}}
\newcommand{\ce}{{\cal E}}
\newcommand{\cf}{{\cal F}}
\newcommand{\ch}{{\cal H}}
\newcommand{\cl}{{\cal L}}
\newcommand{\cm}{{\cal M}}
\newcommand{\cp}{{\cal P}}
\newcommand{\cs}{{\cal S}}
\newcommand{\cz}{{\cal Z}}
\newcommand{\eps}{\varepsilon}
\newcommand{\ra}{\rightarrow}
\newcommand{\la}{\leftarrow}
\newcommand{\Ra}{\Rightarrow}
\newcommand{\dist}{\mbox{\rm dist}}
\newcommand{\bn}{{\mathbf N}}
\newcommand{\bz}{{\mathbf Z}}

\setlength{\parindent}{0pt}
%\setlength{\parskip}{2ex}
\newenvironment{proofof}[1]{\bigskip\noindent{\itshape #1. }}{\hfill$\Box$\medskip}

\renewcommand{\familydefault}{pnc}

\begin{document}

\bigskip

\begin{center}
{\LARGE\bf Assignment-1 of Computer Networking}
\end{center}

\begin{center}
	{\large \bf Liu Yuxi @ \today}
\end{center}

\bigskip

\begin{document}
    \tableofcontents
    \newpage
    \section{Webserver}
    \subsection{Basic target}
    Develop a web server that handles HTTP requests. The web server accepts and parses the HTTP request, gets the requested file from the file system, creates an HTTP response message consisting of the requested file preceded by header lines, and then sends the response directly to the client. If the requested file is not present in the server, the server sends an HTTP "404 Not Found" message back to the client.
    \subsection{Optional target}
    \subsubsection{Multithreaded server}
    Implement a Multithreaded server that is capable of serving muliple requests simultaneously.
    \subsubsection{HTTP client}
    Write a HTTP client to test the server
    \subsection{Implementation}
    \subsubsection{Server}
    \begin{enumerate}[(a)]
        \item Build a socket to bind a adress - (id, post)
        \item Establish a connection by ${.accept()}$ method constantly
        \item Set a threading with target as a function to deal with the connection and arguments as the new socket and new address
        \item Recieve the message by ${.recv()}$ method
        \item Try to open the requested file in the file system
        \item Send HTTP message "HTTP/1.1 200 OK" and the data if the file exists
        \item Send HTTP message "HTTP/1.1 404 Not Found" if the file doesn't exists
    \end{enumerate}
    \subsubsection{Client}
    \begin{enumerate}[(a)]
        \item Connect the server's socket by ${.connect()}$ method
        \item Send "Get /filename HTTP/1.1" to get the file and receive the data
    \end{enumerate}
    \subsection{Conclusion}
    Learn to take advantage of some basic methods of socket and get a good command of the fundamental interactive mechanism of the web server and the client. Besides, realize some formats of HTTP messsages.

    \newpage
    \section{UDPPinger}
    \subsection{Basic target}
    Use UDP sockets to send and receive datagram packets. Learn how to set a proper socket timeout. Compute the statistics such as packet loss rate.
    \subsection{Optional target}
    \subsubsection{RTTs}
    Calculate the round-trip time for each packet and report their minimum, maximum and average. In addition, calculate the packet loss rate.
    \subsubsection{Heartbeat}
    The Heartbeat is used to check if an application is up and running and to report one-way packet loss. The client sends a sequence number and current timestamp in the UDP packet to the server, which is listening for the Heartbeat (i.e., the UDP packets) of the client. Upon receiving the packets, the server calculates the time difference and reports any lost packets. If the Heartbeat packets are missing for some specified period of time, we can assume that the client application has stopped.
    \subsection{Implementation}
    \subsubsection{UDPPingerClient}
    \begin{enumerate}[(a)]
        \item Use ${.settimeout()}$ to set a timeout of each packet
        \item Use ${time()}$ to get the precise time and calculate the time difference the get the RTTs
        \item Use a list to store the RTTs and get the max, min and ave
        \item Write down the number of the "lost"(timeout) packets
    \end{enumerate}
    \subsubsection{UDPHeartbeatServer}
    \begin{enumerate}[(a)]
        \item Receive the packets and write down the time and their sequence number in two lists
        \item Calculate the time difference between the forward packet and the current packet
        \item Print the lost packets between the forward packet and the current packet
    \end{enumerate}
    \subsubsection{UDPHeartbeatClient}
    \begin{enumerate}[(a)]
        \item Send sequence number and local time with a random loss
    \end{enumerate}
    \subsection{Conclusion}
    Learn the fundamental mechanism of UDP and the definition of RTT and packet loss rate.

	\newpage
	\section{MailClient}
	\subsection{Basic target}
	Develop a simple mail client that sends email to any recipient. The client will need to connect to a mail server, dialogue with the mail server using the SMTP protocol, and send an email message to the mail server.
	\subsection{Optional Exercises}
	\subsubsection{TLS or SSL}
	Add a Transport Layer Security(TLS) or Secure Sockets Layer(SSL) for authentication and security reasons
	\subsubsection{Send images}
	Modify the client such that it can send emails with both text and images
	\subsection{Implementation}
	\subsubsection{MailClient}
	\begin{enumerate}[(a)]
		\item Choose ("smtp.qq.com", 587) as the mail server (port 25 is ok without TLS nor SSL)
		\item Follow the rules of SMTP protocol to interact with the server by methods ${.send()}$ and ${.recv()}$
	\end{enumerate}
	\subsubsection{SMTP protocol}
	\begin{enumerate}[(a)]
		\item HELO QQ email
		\item STARTLS
		\item HELO
		\item AUTH LOGIN
		\item (Username)
		\item (Password, actually authorization code in smtp.qq.com)
		\item MAIL FROM : $\langle$ @qq.com $\rangle$
		\item RCPT TO : $\langle$ @ $\rangle$
		\item DATA
		\item Header
		\item MIME-Header
		\item Text-Header
		\item Message data
		\item Image header
		\item Image data
		\item Frontier
		\item QUIT
	\end{enumerate}
	\subsection{Conclusion}
	Learn to interact with the mail server following SMTP protocol. Learn some headers for sending messages both with text and images. Learn the TLS and SSL for security and authentication.

	\newpage
	\section{WebProxyServer}
	\subsection{Basic target}
	Develop a small web proxy server which is able to cache web pages. The proxy server only understands simple GET-requests, but is able to handle all kinds of objects not just HTML pages.
	\subsection{Optional target}
	\subsubsection{404 Not Found}
	When the client requests an object which is not available, send the "404 Not Found" response.
	\subsubsection{POST}
	Add support for POST-request.
	\subsubsection{Caching}
	When the proxy gets a request, it checks if the requested object is cached, and if yes, it returns the object from the cache, without contacting the server. If the object is not cached, the proxy retrieves the object from the server, returns it to the client and caches a copy for future requests.
	\subsection{Implementation}
	\subsubsection{WebProxyServer}
	\begin{enumerate}[(a)]
		\item Establish a cache in the local file system
		\item Establish a tcpSerSocket binded with ('127.0.0.1', 2335) and keep listening
		\item Accept a (tcpCliSocket, addr) from tcpSerSocket.accept()
		\item Receive message and extract the url or filename
		\item Encode the filename(or url) by MD5 and look up the corresponding file in the local cache (Or just the original filename for HelloWorld.html)
		\item If the file exists, proxy server finds a chache hit and generate some response messages(status, header, data)
		\item Catch the IOError when the cache misses and create a new socket on the proxy server
		\item Try to connect the host(created by the url) an port 80
		\item Get the data by GET-message sended to the socket and store the data
		\item Write data down to the local cache and also send it to the tcpCliSocket if successful connection
		\item Otherwise, if the status is '404', send '404 Not Found'
	\end{enumerate}
	\subsection{Conclusion}
	Learn to establish tcp connections to implement a proxy server. Learn some knowledge about different ports and the format of response(header, status, data, etc.). Learn to encode the file by MD5 in the local file system.
\end{document}




















\end{document}
